% Supported aspectratio 43|169, default is 43
\documentclass[aspectratio=169]{beamer}
\usetheme[theme=blue,logo=logowithtextvi]{HUST} % <-- this matches beamerthemeHUST.sty

% \usepackage[utf8]{vietnam}
\usepackage[T5]{fontenc}
% \setmainfont{assets/fonts/LATO-BLACK.TTF}
\usepackage{enumitem}
\usepackage{tcolorbox}
\usepackage{listings}
\usepackage{verbatim}
\usepackage{amsmath}
\tcbuselibrary{listingsutf8}




\newcommand{\placecontent}[4]{%
  \tikz[remember picture,overlay]
    \node[anchor=north west]
      at ([xshift=#1,yshift=-#2]current page.north west)
      {\parbox{#3}{#4}};
}


\title{CẤU TRÚC DỮ LIỆU VÀ THUẬT TOÁN}
\author{SoICT - HUST}
\date{}
\setbeamertemplate{footline}{%
  \hfill%
  \insertframenumber\hspace{0.5cm}\vspace{0.3cm}
  }
\begin{document}

\HUSTInsertBrandSlide
\HUSTInsertThemeSlide

% Title slide - place it here for your customization
{\HUSTUseBackground{onelove.pdf}
\begin{frame}
  \ifdefstring{\insertaspectratio}{169}{
    \HUSTCornerImage{assets/logo/04.pdf}

    %----------You can edit here----------
    \placecontent{0.5cm}{0.33\paperheight}{0.85\paperwidth}{
        \color{\HUSTFrameTitleTextColor}\bfseries\fontsize{22pt}{30pt}\selectfont
        \inserttitle
    }
    % \placecontent{0.5cm}{0.48\paperheight}{10cm}{    % one-line title
    \placecontent{0.5cm}{0.60\paperheight}{0.5\paperwidth}{    % two-line title
        \color{\HUSTFrameTitleTextColor}\fontsize{12pt}{14pt}\selectfont
        TUẦN 1 : CÁC KHÁI NIỆM CƠ BẢN\\
        \insertauthor
    }
  }{}
  \ifdefstring{\insertaspectratio}{43}{
    \HUSTCornerImage[0.1][0.35cm][-0.03]{assets/logo/04.pdf}

    %----------You can edit here----------
    \placecontent{0.5cm}{0.33\paperheight}{0.80\paperwidth}{
        \color{\HUSTFrameTitleTextColor}\bfseries\fontsize{18pt}{22pt}\selectfont
        \inserttitle
    }
    % \placecontent{0.5cm}{0.48\paperheight}{10cm}{    % one-line title
    \placecontent{0.5cm}{0.60\paperheight}{0.5\paperwidth}{    % two-line title
        \color{\HUSTFrameTitleTextColor}\fontsize{12pt}{14pt}\selectfont
        TUẦN 1 : CÁC KHÁI NIỆM CƠ BẢN\\
        \insertauthor
    }
  }{}
\end{frame}
}

% Display the tableofcontents before a section
\AtBeginSection[]
{
    \begin{frame}<beamer>
        \frametitle{MỤC LỤC}
        \tableofcontents[currentsection]
    \end{frame}
}

\begin{frame}{MỤC TIÊU}
    \hspace{1.5cm}\textit{\color{HUSTBlue}Sau bài học này, người học có thể:}\vspace{0.5cm}

    \begin{itemize}
        \item 1. Hiểu được một số \textcolor{HUSTRed}{khái niệm cơ bản về thuật toán}
        \item 2. Biết \textcolor{HUSTRed}{ký hiệu tiệm cận} dùng để đánh giá độ phức tạp thuật toán
        \item 3. Biết cách \textcolor{HUSTRed}{phân tích độ phức tạp của thuật toán}
    \end{itemize}
\end{frame}


\section{Ví dụ minh họa}
\begin{frame}{1. Ví dụ minh họa}
  
  \begin{itemize}[label={\tiny$\blacksquare$}] 
    \item {\color{HUSTBlue}Bài toán tìm dãy con lớn nhất:}
      \begin{itemize}[label={\tiny$\bullet$}]
        \item Cho dãy số gồm $n$ số: $a_0, a_1, a_2, ..., a_{n-1}$

        \item Dãy gồm liên tiếp các số $a_i, a_{i+1},\ldots,a_j$ với $0 \leq i \leq j \leq n-1$ được gọi là 
          \textcolor{HUSTYellow}{dãy con} của dãy đã cho và $\sum_{k=i}^j a_k$ được gọi là trọng lượng của dãy con này
        \item \color{HUSTYellow}\textbf{Hãy tìm trọng lượng lớn nhất của dãy con, tức là tìm cực đại giá trị} $\sum_{k=i}^j a_k$ . Ta gọi dãy con có trọng lượng lớn nhất là \textbf{dãy con lớn nhất.}
    \end{itemize}
    \pause
  \item \textbf{Ví dụ:} Cho dãy số $-2, \textcolor{HUSTYellow}{11, -4, 13}, -5, 2$ thì cần đưa ra câu trả lời là 20 (dãy con lớn nhất là 11, -4, 13 với giá trị $= 11 + (-4) + 13 = 20$)
\end{itemize}
\end{frame}


\begin{frame}[fragile]{1. Ví dụ minh họa}
  \begin{itemize}[label={\tiny$\blacksquare$}]
    \item<1-> {\color{HUSTBlue}Cách 1: Duyệt toàn bộ}
      \begin{itemize}[label={\tiny$\bullet$}]
        \item<1-> Duyệt tất cả các dãy con có thể có của dãy đã cho:
          \textcolor{HUSTYellow}{$a_i, a_{i+1},\ldots,a_j$ với $0 \leq i \leq j \leq n-1$,}
          và tính tổng của mỗi dãy con để tìm ra trọng lượng lớn nhất.
      \end{itemize}
  \end{itemize}

\begin{tcolorbox}[colback=gray!10, colframe=black, title={}]
\begin{lstlisting}[language=C, basicstyle=\ttfamily\footnotesize, breaklines=true]
int maxSum = a[0];
for (int i = 0; i <= n-1; i++) {
    for (int j = i; j <= n-1; j++) {
        int sum = 0;
        for (int k = i; k <= j; k++)
            sum += a[k];
        if (sum > maxSum) maxSum = sum;
    }
}
\end{lstlisting}
\end{tcolorbox}
\end{frame}
  
\begin{frame}[fragile]{1. Ví dụ minh họa}
\begin{itemize}
    \item Cách 1: Duyệt toàn bộ
    \begin{itemize}
        \item Duyệt tất cả các dãy con có thể có của dãy đã cho: $a_i, a_{i+1}, ..., a_j$ với $0 \leq i \leq j \leq n-1$, và tính tổng của mỗi dãy con để tìm ra trọng lượng lớn nhất.
        \item \textbf{Phân tích thuật toán:} Ta sẽ tính số lượng phép cộng mà thuật toán phải thực hiện, tức là đếm xem dòng lệnh \texttt{\textcolor{HUSTYellow}{sum += a[k]}} phải thực hiện bao nhiêu lần. Số lượng phép cộng là:
    \end{itemize}
\end{itemize}

\vspace{0.01cm}

\begin{columns}[T,onlytextwidth]
    \begin{column}{0.5\textwidth}
        \begin{tcolorbox}[colback=gray!10, colframe=black, boxsep=1pt, left=1pt, right=1pt, top=1pt, bottom=1pt, halign=left]
            \tiny
\begin{lstlisting}[language=C, basicstyle=\ttfamily\tiny, breaklines=true]
int maxSum = a[0];
for (int i = 0; i<=n-1; i++) {
for (int j = i; j<=n-1; j++) {
int sum = 0;
for (int k=i; k<=j; k++) sum += a[k];
if (sum > maxSum) maxSum = sum;
    }
}
\end{lstlisting}
        \end{tcolorbox}
    \end{column}

    \begin{column}{0.5\textwidth}
        \scriptsize
        \begin{align*}
            \sum_{i=1}^n \sum_{j=i}^n (j-i+1)
            &= \sum_{i=1}^n [1 + 2 + \cdots + (n-i+1)] \\
            &= \sum_{i=1}^n \frac{(n-i+1)(n-i+2)}{2} \\
            &= \frac{n(n+1)(n+2)}{6}
        \end{align*}
    \end{column}
\end{columns}
\end{frame}

\section{Một số khái niệm cơ bản về thuật toán}
\begin{frame}{2. Một số khái niệm cơ bản về thuật toán}
Nội dung slide
\begin{itemize}
  \item Nội dung 1
  \item Nội dung 2
  \item Nội dung 3
\end{itemize}
\end{frame}

\section{Ký hiệu tiệm cận}
\begin{frame}{3. Ký hiệu tiệm cận}
Nội dung slide
\begin{itemize}
  \item Nội dung 1
  \item Nội dung 2
  \item Nội dung 3
\end{itemize}
\end{frame}

\section{Kỹ thuật phân tích thuật toán}
\begin{frame}{4. Kỹ thuật phân tích thuật toán}
Nội dung slide
\begin{itemize}
  \item Nội dung 1
  \item Nội dung 2
  \item Nội dung 3
\end{itemize}
\end{frame}

{\HUSTUseBackground{theme_hust_oneside.pdf}
\begin{frame}
  \ifdefstring{\insertaspectratio}{169}{
    \placecontent{0.355\paperwidth}{0.410\paperheight}{0.640\paperwidth}{
        \color{HUSTRed}\bfseries\fontsize{28pt}{36pt}\selectfont\centering
        THANK YOU!
    }
  }{}
  \ifdefstring{\insertaspectratio}{43}{
    \placecontent{0.355\paperwidth}{0.440\paperheight}{0.640\paperwidth}{
        \color{HUSTRed}\bfseries\fontsize{28pt}{36pt}\selectfont\centering
        THANK YOU!
    }
  }{}
\end{frame}
}
{\HUSTUseBackground{theme_hust_oneside.pdf}
\begin{frame}
  \ifdefstring{\insertaspectratio}{169}{
    \placecontent{0.355\paperwidth}{0.320\paperheight}{0.640\paperwidth}{
        \color{HUSTRed}\bfseries\fontsize{28pt}{36pt}\selectfont\centering
        CẢM ƠN \\ĐÃ LẮNG NGHE!
    }
  }{}
  \ifdefstring{\insertaspectratio}{43}{
    \placecontent{0.355\paperwidth}{0.350\paperheight}{0.640\paperwidth}{
        \color{HUSTRed}\bfseries\fontsize{28pt}{36pt}\selectfont\centering
        CẢM ƠN \\ĐÃ LẮNG NGHE!
    }
  }{}
\end{frame}
}

\end{document}
